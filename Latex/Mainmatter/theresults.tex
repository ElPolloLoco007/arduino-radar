\chapter{The results}
The complete radar is now serving its goal, but the one thing not working perfectly is the Ultrasonic sensor. There seems to be some kind of error in the code, so when the receiver doesn't receive an echo, the radar glitches for an instant, before continuing on its arc. We discovered that if we used the pre-set library, the glitch did not appear, but that library didn't take in account the temperature. Therefore, we decided to continue with our own code, instead of trying to change the library, also time was becoming an issue.  

The test we made proved to be satisfying, even though there were some slight errors because of the fluctuations of the temperature sensor, these can be seen on the graphs shown earlier. Since the main principle of the radar is working and it's accuracy is precise enough for this project, the radar is found satisfying. 

If we were going to continue working on the radar project, of course fixing the glitch would be our first priority, but adding an OLED display would be very beneficial. Sine the OLED display can give the user a graphical view of the measurements, meaning it looks more like the output you get from an actual radar. Also, it would be highly beneficial to design a surface or box, so that such things as wires, breadboard and Arduino Uno aren't exposed, hence it would look more like a small gadget that you could buy at a local store. It could potentially be a small toy for children, from the age of 10 or so. 
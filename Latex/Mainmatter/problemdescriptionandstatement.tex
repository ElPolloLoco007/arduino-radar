\chapter{Problem description and statement}
The object of this project is to get a small, but basic understanding of Arduino hardware and software, this to be able to create a small working piece of machinery. After some thinking and discussing this machinery is going to be a small radar, which the measurements will be shown on a small screen and also using LED's to indicate an object nearby. Throughout this project there is a learning curve incorporated, where the beginning is a rock bottom, we need to learn Arduino programming and how to utilize it. Thereafter it is possible to start to build the actual project or said in other words, we start with turning on a small LED and expand it into a small-scale radar. Even making some thoughts about how it can be upgraded in the future and possible improvements. 

The radar is built in such a way, that it sends and receives signal in all directions, $360$ degrees, anything else would be useless. However, when creating a small school project, it's a highly simplified version and it has its limitations. The radar has a view range of about $180$ degrees and can only check one degree at the time and is only accurate on static objects.
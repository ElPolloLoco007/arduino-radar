\chapter{How it's done}
Describing the process from start to finish, and the challenges we met along. 

The biggest challenge/uncertainty was to get the components in time. We had to order some components online, since the university did not have them in stock. We ordered the Ultra Sonic, LCD Display and RGB LED module form China. 

So while waiting for these components, we decided to attach and program the other components, which we already had. Like the three LED’s and belonging resistors, the servo motor and the button and to get them to function properly. 

So to start with, we got the LED lighting up with the press of the button and to stay on until the button was pressed again. Adding a second LED, so when the button was pressed on, the green LED ``lighted up'' and when the button was off, the red LED ``light'' up. This button function was then applied to every component throughout the project, more on this later. We then added the Servo motor and got it running smoothly. It was then added to the button function, turning on and off with the LED's. By now the components started arriving from Chine, starting out with the LCD display. Getting this to work properly was a small challenge, especially when rewriting itself, it showed to be challenging to get it to clear itself at the right time. Shortly after the RGB LED and the Ultra Sonic where attached, here the most challenging problem being the Ultra Sonic. The challenge was to get the programming in place for the Ultra Sonic, so that the Servo motor, which it is attached to, still was running smoothly. The Servo motor kept waiting for the Ultra Sonic, since the Ultra Sonic did not receive a signal back and therefore stalled the entire radar and made it not run fluidly.